%\graphicspath{{../images/}}
\section{Обзор методов шумоподавления}
\subsection{Постановка задачи}
Будем считать, что существует исходное изображение $x$, которое не искаженно шумом. После процедуры искажения шумом $n$, получаем шумное изображение $y$. Степень зашумленности изображения можно описать следующим понятием SNR.
SNR (отношение сигнал шум) - безразмерная величина, равная отношению мощности сигнала на мощность шума. В текущих обозначениях получим:
\begin{equation}
	SNR = \frac{P_y}{P_n}
\end{equation} 
где:
\begin{itemize}
	\item $P_x$ - мощность сигнала (изображения)
	\item $P_n$ - мощность шума
\end{itemize}
Значит, что бы уменьшить влияния шума на изображения, необходимо либо увеличить мощность сигнала, что из-за особенностей строения устройств, детектирующих изображение, приведет к увеличению шума. Либо уменьшить количество шума. Вторым методом и занимаются алгоритмы шумоподавления.
В данной ВКР будем считать, что изображение $x$ будет искаженно аддитивным шумом $n$, в результате чего мы получим изображение $y$. Опишем это следующим образом.
\begin{equation}
	y = x + n
\end{equation}
Целью шумоподавления является нахождения  исходного изображения $x$, при этом слагаемое $n$ неизвестно, а доступно лишь только зашумленное изображение $y$.



\subsection{Модели шума}
Шум изображения - это случайное изменение яркости или цветовой информации на изображениях и, как правило, аспект электронного шума. Шум может как зависеть от изображения так и быть независимым. Является нежелательным побочным продуктом захвата изображения, который скрывает желаемую информацию.

Шум изображения может варьироваться от практически незаметных пятен на цифровой фотографии, сделанной при хорошем освещении, до оптических и радиоастрономических изображений, которые почти полностью представляют собой шум, из которого небольшое количество информации может быть получено с помощью сложной обработки. Такой уровень шума был бы недопустим на фотографии, так как было бы невозможно даже определить объект.\cite{Noise}

Характер проблемы удаления шума зависит от типа шума, повреждающего изображение. Рассмотрим некоторые типы шума:
\begin{itemize}
	\item Гауссовский шум
	\item Пуассоновский шум
	\item Импульсный шум
	\item Экспоненциальный шум
\end{itemize}


\subsubsection{Гауссовский  шум}
Модель гауссовского шума является одной из самых популярных моделей шума в задачах шумоподавления, так как он описывает естественные причины появления шумах.
Основные источники гауссовского шума в цифровых изображениях возникают во время получения, например шум датчика, вызванный плохим освещением или высокой температурой.
Гауссовский шум является аддитивным, поэтому процесс искажения цифрового изображения можно описать следующей формулой.
\begin{equation}
	y = x + n
\end{equation}
где:
\begin{itemize}
	\item y - зашумленное изображение 
	\item x - исходное изображение
	\item n - случайная величина имеющая гауссовское распределение.
\end{itemize}
Гауссовский шум можно описать плотностью вероятности имеющий вид:
\begin{equation}
	P(z)=\frac{1}{\sigma\sqrt{2\pi}}e^{\frac{(z-\mu)^2}{2\sigma^2}}
\end{equation}
где
\begin{itemize}
	\item $P(z)$ - плотность вероятности
	\item $z$ - равномерно распределенная случайная величина
	\item $\sigma$ - среднеквадратичное отклонение
	\item $\mu$ - среднее значение
\end{itemize}


\subsubsection{Пуассоновский шум}
Немаловажным на практике является также пуассоновский шум. Он возникает из-за того, что количество фотонов, которые детектируется сенсором не всегда равное количество, даже при одинаковых условиях съемки. Количество фотонов, которые буду задетектированы описываются пуассоновским распределением. Распишем выражения плотности вероятности для него.
\begin{equation}
	P(X=k)=\frac{N^k}{k!}e^{-N}
\end{equation}
где:
\begin{itemize}
	\item $P(X=k)$ - плотность вероятности
	\item $X$ - случайная количество фотонов
	\item $k$ - количество фотонов, которые могут быть детектированы 
	\item $N$ - реальное количество фотонов
\end{itemize}


\subsection{Простейшие алгоритмы}
Для демонстрации работы фильтров будет использовано следующее изображение.
\begin{figure}[H]\label{img:orig}
	\center{\includegraphics[scale=0.7]{C:/Users/nikit/PycharmProjects/diploma/doc/img/ex.jpg}}	
\end{figure}

Всякий раз  мы будем на него накладывать аддитивный гауссовский шум со среднеквадратичным отклонением равным 0.05.
\begin{figure}[H]\label{img:noised}
	\center{\includegraphics[draft, scale=0.7]{C:/Users/nikit/PycharmProjects/diploma/doc/img/exNoised}}
\end{figure}
\subsubsection{Box фильтр}
Первые фильтры были линейными, они были основаны на идее, что пиксели в некоторой малой окрестности имеют примерно одинаковые значения интенсивности. Поэтому если представлять каждый пиксель в виде суммы пикселей в окрестности, то это поможет избавиться от шума. Такой фильтр называется "Box" фильтром. "Box" фильтра задается квадратной матрицей с радиусом r, где каждый элемент матрицы равен $\frac{1}{r^2}$.
\begin{figure}[H]
	\center{\includegraphics[scale=0.7]{kernelBox}}
	\label{img:kernelBox}
	\caption{Пример ядра "box" фильтра с радиусом 3}
\end{figure}
Ниже представлен пример работы фильтра.
\begin{figure}[H]
	\center{\includegraphics[draft]{exBox.png}}
	\caption{Результат применения "box" фильтра с радиусом 5}
\end{figure}
Как можно видеть, результат работы "box"  фильтра имеет артефакты в виде горизонтальных и вертикальных линий. Это является причиной почему данный фильтр не используют на практике.
\subsubsection{Гауссовский фильтр}
Улучшением идеи "Box"  фильтра стал гауссов фильтр. В отличии от ядра "box" фильтра, значения ядра гауссовского фильтра вычисляются с помощью функции гаусса от двух переменных :

\begin{equation}
	G(x,y) = \frac{1}{\sqrt{2\pi}\sigma^2}e^{\frac{x^2+y^2}{2\sigma^2}}
\end{equation}
где
\begin{itemize}
	\item $x,y$ - координаты ядра
	\item $\sigma$ среднеквадратичное отклонение
\end{itemize}

Параметр $\sigma$ обозначает насколько сильно будет размыто изображение, соотвественно от $\sigma$ зависит и радиус ядра фильтра. Т.е. $r=3\sigma$. Ниже представлены графики одномерное функции гаусса с различными значениями $\sigma$.

\begin{figure}[H]
	\center{\includegraphics[draft]{plotGauss.png}}
	\caption{Графики функции гаусса с параметром $\sigma = 3,5,11$}
\end{figure}

Применим фильтр гаусса для рис. \ref{img:orig} с различными параметрами $\sigma$.

\begin{figure}[h]
	\begin{minipage}[h]{0.49\linewidth}
		\center{\includegraphics[draft]{gaussSigma3.png} \\ а)}
	\end{minipage}
	\hfill
	\begin{minipage}[h]{0.49\linewidth}
		\center{\includegraphics[draft]{gaussSigma11.png} \\ б)}
	\end{minipage}
	\caption{Зависимость сигнала от шума для данных.}
	\label{ris:image1}
\end{figure}

\paragraph{Достоинства и недостатки "box" фильтра и фильтра гаусса}
Преимуществом первых фильтров является простая реализация и быстрая скорость работы. Так как процесс шумоподавления, можно представить в виде поэлементного умножения спектрального образа шума на спектральный образ изображения. Но они также обладают одним существенным недостатком. Данные фильтры размывают края, это может негативно сказываться на алгоритмах компьютерного зрения.
\subsubsection{Медианный фильтр}
Избавиться от указанного выше недостатка попытался медианный фильтр. Он основан на той идее, что пиксели в некоторой малой окрестности имеют приблизительно равную интенсивность, а шум, соотвественно сильно отличается. Поэтому для пикселя, для которого вычисляется новое значения, берутся пиксели в некоторой окрестности. Как правило, это квадрат с радиусом r. Данные пиксели сортируются по возрастанию или убыванию и новым значением объявляется то, что находится в середине.

Продемонстрируем на примере работу фильтра.

\begin{figure}[H]
\centering \subfigure[]{\includegraphics[draft]{median3.png}
		\label{img:median3}
	}
\hspace{4ex}
\subfigure[]{\includegraphics[draft]{median11.png}
	\label{img:median11}
}
	\caption{Результат работы медианного фильтра: \subref{img:median3} с радиусом 3; \subref{img:median11} с радиусом 11}
\end{figure}
\paragraph{Достоинства и недостатки медианного фильтра}
Данный алгоритм работает хорошо, только с импульсным шумом, что сильно ограничивает его область использования. Но главным недостатком фильтра является то, что изображение теряет своё визуальное качество, становится более "мультяшным".
