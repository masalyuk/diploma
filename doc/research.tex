\section{Исследования}
\subsection{Тестовое множество}
\subsubsection{Изображения}
Для исследования были взяты изображения из базы ImageNet, а так же из личной коллекции. Была произведенна следующая классификация:
\begin{itemize}
	\item Изображения людей
	\item Изображения архитектуры
	\item Изображения текстуры
	\item Изображения полученные при недостаточной освещенности
\end{itemize}
\subsubsection{Видео}
Для ВКР были взяты видео из youtube8m и так же классифицированы.

\subsection{Схема программы}
Работу реализованной программы можно описать следующим образом. На вход программы подаются данные (видео или изображение). Далее накладывается гауссовский шум. После проводится шумоподавления различными алгоритмами с различными параметрами. В итоге выбирается алгоритм, которые показал наибольшие показатели PSNR и SSIM. Для каждой категории берутся максимальные и средние значения значения PSNR, SSIM.
\begin{figure}[H]
	\center{\includegraphics[draft]{programm.png}}
	\caption{Схема итоговой программы}
\end{figure}
\subsection{Результаты исследования}
Исследование разбито на две части. В первой части для каждого алгоритма подбираются оптимальные параметры, которые дают максимальной значение PSNR или SSIM для определенного типа изображения или видео. Исходя уже из полученных результатов будут сравниваться между собой все алгоритмы.

