ВВЕДЕНИЕ
\subsection{First}
Актуальность: в современном мире методы шумоподавления активно используются для улучшения качества изображений, а так же для их обработки и анализа. Так к примеру уменьшение количества шума позволяет более эффективно использовать методы сегментации изображения или более точных результатов при компьютерной томографии.
\subsection{Second}
Актуальность: изображения полученные с помощью камер или с помощью других устройств всегда будут зашумлены за счёт особенностей строения этих приборов. Поэтому стоит задача уменьшения шума как для эстетических целей, так и для прикладных. Например: для использования более эффективных алгоритмов сегментации изображения или для более точного анализа томографических медицинских изображений и в целом для компьютерного зрения.
\subsection{Goals}
Необходимо проанализировать существующие алгоритмы шумоподавления, их реализация, классификация, выявить плюсы и минусы, определить наиболее приемлимую область применения. Так же попытаться улучшить один из методов.
\subsection{Structure}
Структура работы. Работа состоит из введения, четырех разделов, заключения и списка использованной литературы. В первом разделе  описана модель изображения, шума и типы изображений, взятых для анализа эффективности. Во втором разделе будут подробно рассмотренны несколько из существующих алгоритмов. В третьем разделе будет модифицирован метод основанный на Марковских случайных полях. В четвертом разделе будут сравниваться реализованные алгоритмы. 