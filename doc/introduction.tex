\section*{ВВЕДЕНИЕ}
В результате формировании, передачи и преобразовании с помощью электронных систем, изображения могут 
быть подвержены различным искажениям, что в ряде случаев ухудшает качество изображения, с визуальной
точки зрения, а так же скрывает некоторые участки изображений.
Актуальность задач шумоподавления  изображений, полученных с помощью фотокамер, видеокамер, рентгеновских снимков или другим не искусственным методом, растет с каждым годом по мере развития информационных технологий.

На текущем этапе современных средств компьютерной техники можно выделить несколько направлений.
Распознавание образов - обнаружение на изображении объектов с определенными характеристиками, свойственных некоторому классу объектов. Обработка изображений - преобразует некоторым способом изображение. 
Визуализация - генерирование изображения на основе некоторого описания. Важную роль играют системы 
автоматизации всех этих процессов.

Первостепенной задачей такой системы является улучшение качества изображения. Это
в первую очередь достигается за счет уменьшения количества шума на изображениях. На данный 
момент не существуют универсальных алгоритмов, которые позволят это сделать. Поэтому 
исследования в этой области не прекращаются и по сей день.

Чаще всего шумоподавление служит для улучшения визуального восприятия, но
может также использоваться для каких-то специализированных целей — например, в
медицине для увеличения четкости изображения на рентгеновских снимках, в качестве
предварительной обработки для последующих алгоритмов. Также шумоподавление играет
важную роль при сжатии изображений. При сжатии сильный шум может быть принят за
детали изображения, и это может отрицательно повлиять на результирующее качество.\cite{web:Kalinkina}

Целью данной квалификационной рабты является изучение следующих алгоритмов шумоподавления: BM3D,
Non-Local Means, Guided, Bilateral, Total Variation, Markov Random Field. Так же необходимо
провести сравнительный анализ данных алгоритмов. Дополнительной задачей стоит улучшения алгоритма
основанного на случайных марковских полях.

Дипломная работа состоит из четырех разделов. В первом описаны модели шумов, классификация алгоритмов шумоподавления, описание выборки изображений, взятых для исследования. Во втором разделе подробно описаны изучаемые методы
шумоподавления и подобраны оптимальные параметры для каждого. В третьем разде приводится  описание модификации алгоритма основанного на случайных марковских полях. В четвертом разделе сравниваются алгоритмы.
