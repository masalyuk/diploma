\section{Проблема шумоподавления}
\subsection{Модель шума}
В дипломной работе предпологается, что шум является аддитивным и имеет гауссовское распределение с нулевым средним и конечной дисперсией.
\subsection{Постановка задачи шумоподавления}
Целью методов шумоподавления является восстановление оригинального изображения из зашумленного с минимальной потерей информации о текстуре и краях изображения,
\begin{equation}
	y(i,j) = x(i,j) + n(i,j)
\end{equation}
где $y(i,j)$ - изображения с шумом, $x(i,j)$ - оригинальное изображение, $n(i,j)$ - аддитивный шум с гауссовским распределением, а $i$ и $j$ - координаты пикселей изображения.