\section{Критерии оценки эффективности алгоритмов}
\subsection{PSNR}
Для оценки степени искажения изображений, используются различные методы. Самым популярным является PSNR(pick signal-noise-to-noise ratio). Он обозначает соотношение между максимумом возможного значения изображения и мощностью шума, который искажает данное изображение. Значения лежат в диапазоне от $(0, \infty)$, чем больше PSNR, тем изображение считается близки к оригиналу.
\begin{equation}\label{eq:PSNR}
PSNR = 10log_{10}(\frac{MAX_I^2}{MSE})
\end{equation}
где 
\begin{itemize}
	\item $MAX_I$ - максимально возможное значения пикселя
	\item $MSE = \frac{1}{nm}\sum_{i=0}^{n}\sum_{j=0}^{m}(x_{i,j} - y_{i,j})^2$ - средний квадрат ошибки между оригинальным изображением x и искаженным y.
	\item m,n - высота и ширина изображения
\end{itemize}
PSNR имеет как ряд преимуществ так и некоторые недостатки. К преимуществам относят быстроту и легкость вычисления. К недостаткам: для вычисления необходимо иметь исходную и преобразованную картинку, при этом не может быть точно известно, что  оригинальная картинка не имела искажений, так же в ряде случаев PSNR плохо коррелирует с субъективными мерами оценки  качества. Т.е. высокие значения PSNR не всегда обеспечиваются лучшим качеством картинки с визуальной точки зрения.
\subsection{SSIM}
Из-за несовершенств метрики PSNR разрабатывались все новые и новые методики. Одной из который является SSIM (Structural SIMilarity). Отличительной особенностью метода, является то, что метод учитывает «восприятие ошибки» благодаря учёту структурного изменения информации. Идея заключается в том, что пиксели имеют сильную взаимосвязь, особенно когда они близки пространственно. Данные зависимости несут важную информацию о структуре объектов и о сцене в целом\cite{ssim}.
\begin{equation}\label{eq:SSIM}
SSIM = \frac{(2\mu_x\mu_y+c_1)(2\sigma_{x,y}+c_2)}{(\mu_x^2+\mu_y^2+c_1)(\sigma_x^2+\sigma_y^2+c_2))}
\end{equation}
где
\begin{itemize}
	\item $\mu_x$ - среднее х
	\item $\mu_y$ - среднее y
	\item $\sigma_x^2$ - дисперсия x
	\item $\sigma_y^2$ - дисперсия y
	\item $\sigma_{x,y}^2$ - ковариация x и \item $c_1=(k_1L)^2, c_2=(k_2L)^2$ - две переменных
	\item $L$ - динамический диапазон(максимальное значение пикселя)  
	\item $k_1=0.01, k_2=0.03$ - константы
\end{itemize}
Формула \ref{eq:SSIM} указана только для яркостной компоненты. Значения находятся в отрезке $(-1,1)$, если значение +1, то изображение наиболее похожи. При полном несоответствии -1. Так же данная метрика зависит от размера окна, обычно размер берется $8\times8$.\

\subsubsection{Выбор оптимального алгоритма}
На основе оценок эффективности алгоритмов приведенных выше был предложен следующий метод оценки оптимального алгоритма. Для каждой категории изображений производятся испытания всех реализованных алгоритмов с различными параметрами. Алгоритмы, которые показали максимальные значения SSIM и PSNR для большего количества изображений объявляются оптимальными.

\subsection{Заключение}
Развитие метрик оценивания схожести двух изображений продолжается и по сей день. Охватить их в полном объеме не получится, поэтому для дипломной работы были выбраны наиболее популярные метрики, хоть они и обладают некоторым недостатком. А именно, для них необходимо иметь не зашумленное изображения. Что в целях ВКР не является существенным.
Так же предложен собственный метод выбора оптимального алгоритма, для определенной категории изображений.